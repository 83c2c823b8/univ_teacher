\RequirePackage{luatex85}
\documentclass{ltjsarticle}
\usepackage{luatexja-fontspec}
%\setmainfont[Ligatures=TeX]{MS Mincho}
\setmainjfont[YokoFeatures={JFM=prop},]{UD Digi Kyokasho NK-R}
\usepackage[top=10truemm,bottom=15truemm,left=30truemm,right=30truemm]{geometry}
\usepackage{luatexja} % ltjclasses, ltjsclasses を使うときはこの行不要
\usepackage{amsmath,amssymb,mathtools,ascmac,amsthm,amscd,physics,comment,dcolumn,mypkg,titlesec}
\usepackage[all]{xy}
\renewcommand{\thesubsection}{\arabic{subsection}}
\titleformat*{\section}{\Large}
\setlength{\parindent}{0pt}
\pagestyle{empty}
%\everymath{\displaystyle}
\begin{document}
\begin{center}
  \Large{学習指導案}
\end{center}
\hfill 指導教諭:\\
\hfill 授業者:大阪大学理学部 山本雄大\\
\hfill 印\\
日時:\\
場所:\\
対象:\\
\subsection{単元名}\vspace*{-3mm}
  \(x^2\)に比例する関数
  \vspace*{-3mm}
\subsection{単元の目標}\vspace*{-3mm}
  \(x^2\)に比例する関数のグラフをイメージし,その特徴を理解できるようになる.また,そのグラフを用いて問題を解けるようになる.
  \vspace*{-3mm}
\subsection{単元の評価規準}\vspace*{-3mm}
\subsubsection*{(1)知識・技能}
  日常の事象のなかには,\(y=ax^2\)として捉えることができるものがあることを知る.
  また,1次関数や2次関数以外にも関係で関数となっているものが存在することを知る.
  \vspace*{-3mm}
\subsubsection*{(2)思考判断表現}
  \(y=ax^2\)となっていると見なせるものに対して,
  表,式,グラフを用いて表現し考察できる.
  \vspace*{-3mm}
\subsubsection*{(3)主体的に学習に取り組む態度}
  日常の事象などからヒントを得て,抽象化・一般化するなどの数学的な捉え方をすることができる.
  \vspace*{-3mm}
\subsection{単元について}\vspace*{-3mm}
  \subsubsection*{(1)教材観}
    生徒たちは,1年次で比例や反比例について学び,そこで2つの数値の集まりを
    表,式,グラフを用いて表せられること,2年次には,連立方程式と
    1次関数とそのグラフとの関係を学んできている.そこで,今まで学んできた
    関数と比較して,1次関数は変化率が定義域から選択してきた2点に依らず一定であったのに対し,
    2次関数はそうでないことなどの相違点,\(x\)を決めると\(y\)が一つに定まるという関数に
    なっているという点などの共通点を学ぶ.
  \subsubsection*{(2)生徒観}
    数学の得意としている生徒がいる一方で,
    連立方程式などの数値計算はできるものの,関数の概念の理解が追いついておらず
    グラフとの関係などが分かっていない生徒も存在する.
  \subsubsection*{(3)指導観}
    関数の概念が分かっていない生徒もいるため,関数の概念とグラフとの関係を意識した伝え方をし,理解が伴っているか
    確認する.また,具体的な2次関数を提示しそのグラフを自ら描いてもらい,2次関数のイメージをもち,
    問題を解くときには,グラフなどの視覚的な情報から解き方などの推論ができるように伝える.
\subsection{単元の指導計画}\vspace*{-3mm}

\subsection{本時の指導計画}\vspace*{-3mm}
  \subsubsection*{(1)目標}
    放物線のおおまかなイメージを持ち,\(y=ax^2\)の係数\(a\)の変化に伴うグラフの変化を理解し,
    与えられた式から概形を描けるようになる.
  \subsubsection*{(2)使用教材,準備物など}
    教科書,プリント.
  \subsubsection*{(3)展開計画}\vspace*{3mm}
  \begin{center}
    \scalebox{1}[1]{
      \begin{tabular}{|c|c|c|c|c|} \hline
        時間 & 指導内容 & 生徒の学習活動 & 指導上の留意点 & 評価\\ \hline
        5分& 1次関数の復習 & \begin{tabular}{l}1次関数のグラフを書\\いてみて, 関数とグ\\ラフの関係を思い出す.\end{tabular}&  & \begin{tabular}{l}\(x\)に対して\(y\)が一つ定\\まるという関数の概念\\が理解できているか.\\1次関数のグラフが正\\しく描けているか.\end{tabular} \\ \hline
        10分 & 導入1 & \begin{tabular}{l}関数\(y=x^2\)について,\\配布した表とを用いて\\座標平面上に\(x\)が整数\\の点についてプロット\\し,黒板で班ごとに発表する.\end{tabular}& a & a \\ \hline
        10分 & 導入2 & \begin{tabular}{l}\(\frac12\)間隔で表を作成し,\\それを座標平面上にプ\\ロットし,繋げてみる.\end{tabular} & & \begin{tabular}{l}2乗の計算ができて,\\ \(y\)軸に対称か,\(|x|\)に関し\\て増加しているか.\end{tabular}\\ \hline
        15分 & 展開 & \begin{tabular}{l}\(y=-x^2,\ y=\frac{1}{2}x^2,\ y= 2x^2\)\\ を導入時と同様に表からグラフ\\を作成する.\end{tabular}& &  \\ \hline
        5分 & まとめ & \begin{tabular}{l}本時で学んだ内容の係\\数\(a\)におけるグラフの\\概形の変化や違いにつ\\いて確認する.\end{tabular}& & \\\hline
      \end{tabular}}
  \end{center}

  \subsubsection*{(4)板書計画}
\end{document}