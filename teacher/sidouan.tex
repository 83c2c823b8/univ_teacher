\RequirePackage{luatex85}
\documentclass{ltjsarticle}
\usepackage{luatexja-fontspec}
%\setmainfont[Ligatures=TeX]{MS Mincho}
\setmainjfont[YokoFeatures={JFM=prop},]{UD Digi Kyokasho NK-R}
\usepackage[top=10truemm,bottom=10truemm,left=20truemm,right=20truemm]{geometry}
\usepackage{luatexja} % ltjclasses, ltjsclasses を使うときはこの行不要
\usepackage{amsmath,amssymb,mathtools,ascmac,amsthm,amscd,physics,comment,dcolumn,mypkg,titlesec}
\usepackage[all]{xy}
\renewcommand{\thesubsection}{\arabic{subsection}}
\titleformat*{\section}{\Large}
\setlength{\parindent}{0pt}
\pagestyle{empty}
%\everymath{\displaystyle}
\begin{document}
\begin{center}
  \Large{学習指導案}
\end{center}
\hfill 指導教諭:\\
\hfill 授業者:大阪大学理学部 山本雄大\\
\hfill 印\\
日時:\\
場所:\\
対象:\\
\subsection{単元名}
  \(x^2\)に比例する関数
\subsection{単元の目標}

\subsection{単元の評価規準}
\subsubsection*{(1)知識・技能}
  \(\)
\subsubsection*{(2)思考判断表現}

\subsubsection*{(3)主体的に学習に取り組む態度}

\subsection{単元について}
  \subsubsection*{(1)教材観}

  \subsubsection*{(2)生徒観}

  \subsubsection*{(3)指導観}

\subsection{単元の指導計画}
\subsection{本時の指導計画}

\end{document}