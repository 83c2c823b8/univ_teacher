\RequirePackage{luatex85}
\documentclass{ltjsarticle}
\usepackage[top=10truemm,bottom=10truemm,left=20truemm,right=20truemm]{geometry}
\usepackage{luatexja} % ltjclasses, ltjsclasses を使うときはこの行不要
\usepackage{amsmath,amssymb,mathtools,ascmac,amsthm,amscd,physics,comment,mypkg}
\usepackage[all]{xy}
\pagestyle{empty}
%\everymath{\displaystyle}
\geometry{top=10mm}
\begin{document}
{\textbf{\LARGE{数学科教育法Ⅱ最終レポート}}}\hspace{\fill} {\texttt{\Large{04A21051}}}{\Large{山本 雄大}}\\\\
\par 
主体的な学びを生徒たちが教師からの数学の情報を単に受け入れるだけでなく自分自身で問題を設定して解決策を模索して
個々人のペースで考えを深める活動,対話的な学びを生徒同士や教師と話し合うことを通じてアイデア,考え方や発想を
共有することで互いに新たな視点や理解を取り入れる活動,深い学びを単に事実を覚えるだけでなく,その背後にある原理や理由など
も込めて一般的に考えたり応用出来たりする学びであると定義します.\par
まず最初に毎回の授業で「主体的・対話的で深い学び」を実現する方法
について考えたいと思います.主体的な学びは生徒観も鑑みながら,教科書の範疇を
多少超えるような問いかけや事実などを述べるなどして,その証明や何の一般化になっているかを考えてみる活動を促すことが期待されます.
例えば,三平方の定理の証明などを一通り終えたときに中線定理がなりたつことを示唆するといったことが挙げられます.対話的,深い学びを実現する
方法としてはペアやグループなどに分かれてそれぞれ異なる問題や与えられた課題について考え,理解し終えたら互いに説明し合うという機会を設けることが
考えられます.この生徒たちは活動によって他人に説明するために疑問点の解消を目指すことで,理由なども込めた理解が促進されると考えました.\par
次に特に重点的に実施するときの方法については,数学の歴史などを調べその中で生徒が興味をもった事柄の数学的側面について調べ,


\end{document}