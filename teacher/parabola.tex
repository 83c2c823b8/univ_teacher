\RequirePackage{luatex85}
\documentclass[a3papaer,landscape,twocolumn]{ltjsarticle}
\usepackage{luatexja-fontspec}
%\setmainfont[Ligatures=TeX]{MS Mincho}
\setmainjfont[YokoFeatures={JFM=prop},]{UD Digi Kyokasho NK-R}
\usepackage[top=10truemm,bottom=10truemm,left=20truemm,right=20truemm]{geometry}
\usepackage{luatexja} % ltjclasses, ltjsclasses を使うときはこの行不要
\usepackage{amsmath,amssymb,mathtools,ascmac,amsthm,amscd,physics,comment,dcolumn,mypkg,titlesec,multicol,tikz}
\usepackage[all]{xy}
\setlength{\columnseprule}{0.2pt}
\titleformat*{\section}{\LARGE\bfseries}
\setlength{\parindent}{0pt}
\pagestyle{empty}
%\everymath{\displaystyle}
\begin{document}
%{\textbf{\LARGE{ }\ \Large{No.}}}\hspace{\fill} {\texttt{\Large{04A21051}}}\\
\twocolumn[{\Large{4章\ 関数\ \(y=ax^2\)} }\hfill (06/07/23)\\ \Large{\S 1 関数とグラフ}\\]
\section*{1次関数の復習}
\(y= -x + 1\)のグラフを書いてみよう.
\begin{center}
  \begin{tabular}{c|c c c c c c c c c c} \hline
    \(x\) &\(\cdots\) & -3 & -2 & -1 & 0 & 1 & 2 & 3 & \(\cdots\) &\\ \hline
    \(y\) &           &    &    &    &   &   &   &   &            &\\ \hline
  \end{tabular}\\[10pt]

  \begin{tikzpicture}
    \draw[dotted,thin,step=0.5] (-3,-3) grid (3,3);
    \draw [->,thick] (-3.5,0) -- (3.5,0) node [right]{\(x\)};
    \draw [->,thick] (0,-3.5) -- (0,3.5) node [above]{\(y\)};
    \node [below left] (0,0){O};
    \end{tikzpicture}
\end{center}
\leadsto 関数とは\(x\)を定めるごとに\(y\)がただひとつに決まるものであった.\newpage

\section*{\(y=x^2\)のグラフ}
1次関数と同様に\(y=x^2\)の表を完成させて,\(x\)と\(y\)の値の組を座標とする点を書き入れて点をつなぐことで概形を考えてみよう.
\begin{center}
  \begin{tabular}{c|c c c c c c c c c c} \hline
    \(x\) &\(\cdots\) & -3 & -2 & -1 & 0 & 1 & 2 & 3 & \(\cdots\) &\\ \hline
    \(y\) &           &    &    &    &   &   &   &   &            &\\ \hline
  \end{tabular}\\[10pt]

  \begin{tikzpicture}
    \draw[dotted,thin,step=0.5] (-3,-0.5) grid (3,8.5);
    \draw [->,thick] (-3.5,0) -- (3.5,0) node [right]{\(x\)};
    \draw [->,thick] (0,-1) -- (0,8.5) node [above]{\(y\)};
    \node [below left] (0,0){O};
    \draw(1.5,0)node[below]{\(3\)};
    \draw(-1.5,0)node[below]{\(-3\)};
    \draw(0,2.5)node[left]{\(5\)};
    \draw(0,5)node[left]{\(10\)};
    \draw(0,7.5)node[left]{\(15\)};
    \end{tikzpicture}
\end{center}
\newpage

点を取っていないところを確認するため,原点近くのグラフの様子を調べてみよう.\\[15pt]
\begin{center}
  \begin{tabular}{c|c c c c c c c c c ccccccccccccccc} \hline
    \(x\)  & -1 & -0.9 & -0.8 & -0.7 & -0.6 & -0.5 & -0.4 & -0.3 & -0.2 & -0.1  \\ \hline
    \(y\) &     &    &    &    &   &   &   &   &            &\\ \hline
  \end{tabular}\\[5pt]
  \begin{tabular}{c|c c c c c c c c c ccccccccccccccc} \hline
    \(x\)   & 0 & 0.1 & 0.2 & 0.3 & 0.4 & 0.5 & 0.6 & 0.7 & 0.8 & 0.9 & 1 \\ \hline
    \(y\) &     &    &    &    &   &   &   &   &            &\\ \hline
  \end{tabular}\\[30pt]

  \begin{tikzpicture}
    \draw[dotted,thin,step=0.5] (-5.5,-0.5) grid (5.5,5.5);
    \draw [->,thick] (-5.5,0) -- (5.5,0) node [right]{\(x\)};
    \draw [->,thick] (0,-1) -- (0,5.5) node [above]{\(y\)};
    \node [below left] (0,0){O};
    \draw(2.5,0)node[below]{\(0.5\)};
    \draw(-2.5,0)node[below]{\(-0.5\)};
    \draw(5,0)node[below]{\(1\)};
    \draw(-5,0)node[below]{\(-1\)};
    \draw(0,2.5)node[left]{\(0.5\)};
    \draw(0,5)node[left]{\(1\)};
  \end{tikzpicture} 
\end{center}

\newpage
  \section*{わかったこと,気付いたこと}
  \begin{itemize}
   \item 
   \item \vspace*{10mm}
   \item \vspace*{10mm}
  \end{itemize}
  
\end{document}