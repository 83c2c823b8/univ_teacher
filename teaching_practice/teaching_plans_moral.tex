\RequirePackage{luatex85}
\documentclass[leqno]{ltjsarticle}% leqnoは数式番号左
\usepackage{luatexja-fontspec}
\usepackage[top=5truemm,bottom=5truemm,left=25truemm,right=25truemm]{geometry}
\usepackage{luatexja} 
\usepackage{multicol,multirow,amsmath,amssymb,mathtools,ascmac,amsthm,amscd,physics,comment,dcolumn,titlesec,mathrsfs,longtable,tabularx,mypkg}
\usepackage[all]{xy}
\titleformat*{\section}{\Large\bfseries}
\setlength{\parindent}{0pt}
\pagestyle{empty}
%\everymath{\displaystyle}
\begin{document}
{\center{{\textbf{\large{包む}}}}
\begin{flushright}
	令和06年06月10日(水)第2校時\\
	場所 2年2組教室\\
	授業者\ \ \ 山本 雄大
\end{flushright}
\begin{table}[htbp]
	\begin{tabular}{|p{8em}|>{\raggedright}p{17em}|>{\raggedright\arraybackslash}p{17em}|}
		\hline
		学習内容 & 生徒の学習活動 & 教師の支援(・)・評価(※)・協働(◇) \\
   \hline \hline
		1.導入&  & ・風呂敷を見て,知っているか尋ねる.\tabularnewline
	\hline

	2. 展開& 教材「包む」を読み,考える. & ・読み上げる.  \tabularnewline
				 &&\tabularnewline

				 & ・発問 「風呂敷の良いところを考えてみよう.」\\ &・班ごとにあがったを利点を黒板に書く.  \tabularnewline
				 & 班になる.& \tabularnewline
				 & 隣同士で風呂敷の真結びをして,かんたんバッグを作る.\\
				  資料の中から包みたいものを自由に作る.
				 &・風呂敷とその包み方のプリントを班ごとに2枚ずつ配る. \tabularnewline
	 &  & \tabularnewline
	
	 &  & \tabularnewline 
		

	 &風呂敷で  & \tabularnewline

	%& ・発問 「作ってみて感じたこと.」\\ 「受け渡しをして感じたこと.」& \tabularnewline
	
	  & &\tabularnewline

	 & & \tabularnewline
		
	 & & \tabularnewline
\hline


 3.&  & \tabularnewline
		
		 \hline

	\end{tabular}
\end{table}

\end{document}}
