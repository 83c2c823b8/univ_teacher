\RequirePackage{luatex85}
\documentclass[leqno]{ltjsarticle}% leqnoは数式番号左
\usepackage{luatexja-fontspec}
\usepackage[top=5truemm,bottom=5truemm,left=25truemm,right=25truemm]{geometry}
\usepackage{luatexja} 
\usepackage{multicol,multirow,amsmath,amssymb,mathtools,ascmac,amsthm,amscd,physics,comment,dcolumn,titlesec,mathrsfs,longtable,tabularx,mypkg}
\usepackage[all]{xy}
\titleformat*{\section}{\Large\bfseries}
\setlength{\parindent}{0pt}
\pagestyle{empty}
%\everymath{\displaystyle}
\begin{document}
{\center{{\textbf{\large{包む}}}}
\begin{flushright}
	令和06年06月10日(水)第2校時\\
	場所 2年2組教室\\
	授業者\ \ \ 山本 雄大
\end{flushright}
\begin{table}[htbp]
	\begin{tabular}{|p{8em}|>{\raggedright}p{17em}|>{\raggedright\arraybackslash}p{17em}|}
		\hline
		学習内容 & 生徒の学習活動 & 教師の支援(・)・評価(※)・協働(◇) \\
   \hline \hline
		1.導入(5分)& & ・風呂敷を見て,知っているか尋ねる.何に使うかなど\tabularnewline
	\hline

	2. 展開(30分) &教科書p.180\\ 教材「包む」を読み,考える.(5分) & ・読み上げる.  \tabularnewline
				 & &\tabularnewline

				 & ・発問 「風呂敷の良いところを考えてみよう.」(5分) ノートp.35\\ &\tabularnewline
				 & 班になる.& \tabularnewline
				 & 班ごとに意見を出す.(5分)&◇ 生徒同士で意見交換する.\tabularnewline
				 
				 & &・班ごとにあがったを利点を黒板に書く.\tabularnewline

				 &・風呂敷2枚とその包み方のプリントを班ごとに配る.  &\tabularnewline

				 & 隣同士で風呂敷の真結びの練習をする.(5分) & ・真結びができているか見て回る.\tabularnewline

				 & 筆箱,教科書などを包んでみて渡してみる.(10分)& ◇生徒同士で渡し合う.  \tabularnewline

				 &  & \tabularnewline 
				 \hline

	3.終末(のこり) & 包んで渡されたときに,どのように感じたかをノートに書く.(5分)& \tabularnewline

				 & & \tabularnewline

				 & 資料の中から包みたいものを自由に作る. \begin{screen} \begin{itemize}
				 	\item ティッシュケース
					\item ペットボトルホルダー
					\item ブックカバー 
				 \end{itemize} \end{screen}&\tabularnewline

	
		

				 &  & \tabularnewline

	%& ・発問 「作ってみて感じたこと.」\\ 「受け渡しをして感じたこと.」& \tabularnewline
	
	  & &\tabularnewline

	 & & \tabularnewline
		
	 & & \tabularnewline
\hline


		
		 \hline

	\end{tabular}
\end{table}

\end{document}}
