\RequirePackage{luatex85}
\documentclass[b4paper,16.5pt,paparesize, landscape]{ltjsarticle}% leqnoは数式番号左
\usepackage{luatexja-fontspec}
\usepackage[top=10truemm,bottom=10truemm,left=10truemm,right=10truemm]{geometry}
\usepackage{luatexja} 
\usepackage{multicol,amsmath,amssymb,mathtools,ascmac,amsthm,amscd,physics,comment,dcolumn,titlesec,mathrsfs,mypkg,cases,cleveref}
\usepackage[all]{xy}
\titleformat*{\section}{\Large\bfseries}
\setlength{\parindent}{0pt}
\pagestyle{empty}
\everymath{\displaystyle}
\setlength{\columnseprule}{0.4pt}
\begin{document}
{\textbf{\Large{加減法}}}\hspace{\fill}{\scalebox{1.5}{( )組(        )}}\\
\begin{multicols*}{3}
	\begin{screen}
		\vspace{5mm}
	\begin{enumerate}
	 \item[Step1]
			片方の式を整数倍して\underbar{\hspace{20mm}}の絶対値をそろえる.\\
	 \item[Step2]
			足すか引くかして文字を\underbar{\hspace{20mm}}する.\\
	 \item[Step3]
		 解く!\\
	 \item[Step4]
		 \underbar{\hspace{20mm}}して解く.\\
	 \item[Step5]
		 確かめる.\\
\end{enumerate}
	\end{screen}
	\\[10pt]
	例)
	\begin{numcases}
		{}
		x - 3y = -5 \label{first}\\
		2x -4y = -4 \label{second}
	\end{numcases}
\begin{enumerate}
	\item[Step1]\hfill\\
		(\ref{first})$\times$ 2\\
		\begin{equation}
			2x - 6y = -10
		\end{equation}
	\item[Step2]\hfill\\
		(3) $ - $ (2)
			\begin{equation*}
				-2y = -6
			\end{equation*}
	\item[Step3]\hfill\\
		\begin{equation*}
			y = 3
		\end{equation*}
	\item[Step4]\hfill\\
		(1)に$y=3$を代入して,
		\begin{gather*}
			x - 9 = -5\\
			x = 4
		\end{gather*}
	\item[Step5]\hfill\\	
		$(x,y) = (4,3)$を(1),(2)の左辺に代入するとそれぞれ
		\begin{eqnarray*}
			4 - 3\times 3 = -5 \\
			2\times 4 - 4\times 3 = -4
		\end{eqnarray*}
		右辺と一致するからOK
		
\end{enumerate}
\vfill\null\columnbreak

$(1)\begin{cases}
			x + 2y  = 3\\
			3x + 4y = 7\\
			%(1,1)
\end{cases}$ 	\\[90mm]
(2)$\begin{cases}
			2x  - 5y = 7\\
			6x + 4y = 2\\
			%(1,-1)
\end{cases}$


 \vfill\null\columnbreak
 \begin{itembox}[l]{(3)}
	 \vspace{25mm}
 \end{itembox}

\end{multicols*}
\end{document}
