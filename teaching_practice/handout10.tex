\RequirePackage{luatex85}
\documentclass[b4paper,16.5pt,paparesize, landscape]{ltjsarticle}% leqnoは数式番号左
\usepackage{luatexja-fontspec}
\usepackage[top=10truemm,bottom=10truemm,left=10truemm,right=10truemm]{geometry}
\usepackage{luatexja} 
\usepackage{multicol,amsmath,amssymb,mathtools,ascmac,amsthm,amscd,physics,comment,dcolumn,titlesec,mathrsfs,mypkg}
\usepackage[all]{xy}
\titleformat*{\section}{\Large\bfseries}
\setlength{\parindent}{0pt}
\pagestyle{empty}
\everymath{\displaystyle}
\setlength{\columnseprule}{0.4pt}
\begin{document}
{\textbf{\Large{加減法}}}\hspace{\fill}{\scalebox{1.5}{( )組(        )}}\\
\begin{multicols*}{3}
$(1)\begin{cases}
			x + 2y  = 3\\
			x + 3y = 5\\
\end{cases}$ 	\\[82mm]
(2)$\begin{cases}
			9x + 4y = -15\\
			7x - 4y = -26\\
\end{cases}$\vfill\null\columnbreak

$(3)\begin{cases}
			3x  - 5y = 13\\
			x + 7y = -13\\
\end{cases}$ 	\\[82mm]
$(4)\begin{cases}
				-7x - 8y = 3\\
				14x + 5y = 27\\
\end{cases}$
 \vfill\null\columnbreak
 \begin{enumerate}
	 \item[Step1]
		どちらかの式を整数倍して,消去したい文字の係数の絶対値を揃える.\\
	 \item[Step2]
	 \item[Step3]
	 \item[Step4]
	 \item[Step5]
 \end{enumerate}


\end{multicols*}
\end{document}
