\RequirePackage{luatex85}
\documentclass[b4paper,16.5pt,paparesize, landscape]{ltjsarticle}% leqnoは数式番号左
\usepackage{luatexja-fontspec}
\usepackage[top=10truemm,bottom=10truemm,left=10truemm,right=10truemm]{geometry}
\usepackage{luatexja} 
\usepackage{multicol,amsmath,amssymb,mathtools,ascmac,amsthm,amscd,physics,comment,dcolumn,titlesec,mathrsfs,mypkg}
\usepackage[all]{xy}
\titleformat*{\section}{\Large\bfseries}
\setlength{\parindent}{0pt}
\pagestyle{empty}
%\everymath{\displaystyle}
\setlength{\columnseprule}{0.4pt}
\begin{document}
{\textbf{\Large{連立方程式の利用}}}\hspace{\fill}{\scalebox{1.5}{( )組(        )}}\\
\begin{multicols*}{3}
	\begin{itembox}[l]{復習}
ケーキ4個と150円のジュース1本買うとおつりは1550円でした.ケーキ1個はいくらですか?
\end{itembox}
 \vfill\null\columnbreak

\toi{1}{
	1本100円のボールペンと1個150円の消しゴムを合わせて10個買うと1200円でした.それぞれいくつ買いましたか.
} \vfill\null\columnbreak

\toi{2}{
あるレジャー施設の入園料は,おとな2人と小学生1人で4500円\\
大人1人と小学生2人で3600円でした.
おとな1人と小学生1人の入園料をそれぞれ求めなさい.
}

\vfill\null\columnbreak

\toi{3}{
	2つの数の和が59で差が37のとき,この2つの数を求めよ.
}
\vfill\null\columnbreak

\toi{4}{
	
}
\vfill\null\columnbreak

\toi{5}{
}
\vfill\null\columnbreak

\end{multicols*}

\end{document}
