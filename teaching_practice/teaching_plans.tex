\RequirePackage{luatex85}
\documentclass[leqno]{ltjsarticle}% leqnoは数式番号左
\usepackage{luatexja-fontspec}
\usepackage[top=20truemm,bottom=20truemm,left=25truemm,right=25truemm]{geometry}
\usepackage{luatexja} 
\usepackage{multicol,multirow,amsmath,amssymb,mathtools,ascmac,amsthm,amscd,physics,comment,dcolumn,titlesec,mathrsfs,longtable,tabularx,mypkg}
\usepackage[all]{xy}
\titleformat*{\section}{\Large\bfseries}
\setlength{\parindent}{0pt}
\pagestyle{empty}
%\everymath{\displaystyle}
\begin{document}
{\center{{\textbf{\large{第2学年2組 数学科学習指導案}}}}
\begin{flushright}
	令和06年06月03日(月)第2校時\\
	場所 2年2組教室\\
	授業者\ \ \ 山本 雄大
\end{flushright}
\begin{enumerate}
	\item[1]
		単元(題材・主題)連立方程式
	\item[2] 単元(題材)について

		\hspace{1em}二元一次方程式には解$(x,y)$が多くあり,2つの本質的に異なる方程式が存在すると
		その方程式を同時に満たす解が一意に定まり,それを求めるということが連立方程式である
		ことを学ぶ.
		そこで連立方程式の解き方として加減法と代入法の2種類があるが
		双方とも未知数を1つ消去して一元一次方程式に帰着していることを認識し,
		その上で係数などをみて代入法と加減法の上手な使い分けができるようになることを
		狙いとしている.\hfill
		

	\item[3]指導について

		\hspace{1em}第1学年の数と式において一元一次方程式では未知数$x$などを用いて
		数量を文字を使ってあらわすことや項,係数,代入などの用語も学ん
		でいる.一元一次方程式の解き方を念頭において,それに帰着するために
		代入法,加減法を用いて解いていく.
		一元一次方程式の2年生の一次関数の単元に入ったときに連立方程式を解くことは
		二直線の交点を求めていたということで2つ式があると1つの解が定まることに
		納得してほしい.連立方程式の利用の節では日常の種々の問題での未知数を$(x,y)$
		などとおいて方程式を解くということに帰着することで問題を容易に捉えられるように
		することを体感させたい.\\
		\hspace{1em}本学級は
		


	\item[4]	本時の目標
		\begin{enumerate}
			\item[(1)]
				文章から未知数2つを設定して連立方程式に落とし込むことができる.
				
			\item[(2)]
				
				
				
		\end{enumerate}
	\item[5]
		本時の指導過程 \begin{table}[htbp]
			\centering
	\begin{tabular}{|p{8em}|>{\raggedright}p{17em}|>{\raggedright\arraybackslash}p{17em}|}
			 \hline
		学習内容 & 生徒の学習活動 & 教師の支援(・)・評価(※)・協働(◇) \\
     & &   \\
   \hline \hline
		1 本時の目標の確認 & & \tabularnewline
		\multicolumn{3}{|c|}{\fbox{文章から$x,y$などと文字をおいて連立方程式を設定できるようになろう.} }\tabularnewline

				2 既習内容の確認 & ・復習問題を解く.\\ 連立方程式の解き方の確認\\ 1年生の方程式の利用 &   \tabularnewline
													&	
			$\begin{cases}x + y = 6 \\ 2x + y = 10\end{cases} \ 
		\begin{cases}3x - 2y = 1 \\ 6x - 5y = -2\end{cases} $\\ 
													& \tabularnewline
\multicolumn{3}{|c|}{
		\toi{1}{2000円でケーキ4個と150円のジュースを1本買うとおつりが450円でした.ケーキ一個はいくらですか.}
		}\tabularnewline

		& ・方程式をたてて解く手順を確認する.\\ 
			(1) 不明な量を文字で置く.\\
			(2) 式をたてる.\\
			(3) 一次方程式を解く.\\ &\tabularnewline

		3課題提示&・連立方程式の利用の課題に取り組む.\\
		一次方程式でも簡単に解ける問題		&  \tabularnewline

						& & \tabularnewline
		\multicolumn{3}{|c|}{
				\toi{2}{
					1本100円のボールペンと1個150円の消しゴムを合わせて10個買うと1200円でした.
					それぞれいくつ買いましたか.
				}
		}\tabularnewline
		&  ・一次方程式を用いる解法 \\ 
		ボールペン$x$本買ったとおいて
		$  100x + 150(10 - x) = 1200 $ \\  & 
		・解き終えた生徒に他の解法を考えさせる.\tabularnewline
		&  ・連立方程式を用いる解法 \\ 
		ボールペンと消しゴムの買った数をそれぞれ$x,y$とおいて \\
		$\begin{cases}
		100x + 150y = 1200\\
		x + y = 10
	\end{cases}$\\[1em] 
			$(x,y) = (6,4)$& \tabularnewline
   \hline
		\end{tabular}
		
	\end{table}
	
		
\newpage
\begin{table}[htbp]
	\begin{tabular}{|p{8em}|>{\raggedright}p{17em}|>{\raggedright\arraybackslash}p{17em}|}
		\hline 
		& & \tabularnewline

		\multicolumn{3}{|c|}{
		\toi{3}{
		あるレジャー施設の入園料は,おとな2人と小学生1人で4500円\\
		大人1人と小学生2人で3600円でした.
		おとな1人と小学生1人の入園料をそれぞれ求めなさい.
		}
	}\tabularnewline

		4練習問題 &   &  \tabularnewline 
		
    5&  & \tabularnewline
		
		6& &  \tabularnewline
		
		7& &  \tabularnewline
		
		\hline
	\end{tabular}
\end{table}
	\item[6]
		評価の観点
		
\end{enumerate}
\end{document}}
