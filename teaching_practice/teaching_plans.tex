\RequirePackage{luatex85}
\documentclass[11pt]{ltjsarticle}% leqnoは数式番号左
\usepackage{luatexja-fontspec}
\usepackage[top=20truemm,bottom=10truemm,left=25truemm,right=25truemm]{geometry}
\usepackage{luatexja} 
\usepackage{multicol,multirow,amsmath,amssymb,mathtools,ascmac,amsthm,amscd,physics,comment,dcolumn,titlesec,mathrsfs,longtable,tabularx,mypkg}
\usepackage[all]{xy}
\titleformat*{\section}{\Large\bfseries}
\setlength{\parindent}{0pt}
\pagestyle{empty}
%\everymath{\displaystyle}
\begin{document}
{\center{{\textbf{\large{第2学年2組 数学科学習指導案}}}}
\begin{flushright}
	令和06年06月20日(木)第1校時\\
	場所 2年2組教室\\
	授業者\ \ \ 山本 雄大
\end{flushright}
\begin{enumerate}
	\item[1]
		単元(題材・主題)\hspace{5mm}連立方程式の利用
		\vspace{5mm}
	\item[2] 単元(題材)について

	\hspace{1em}
	二元一次方程式には解が多くあり,2つの異なる二元一次方程式を同時に満たす解は
	一意に定まる.それを求めることが連立方程式であることを学んでいく.連立方程式
	の解き方として加減法と代入法の2種類があるが,双方とも未知数を1つ消去して一元一次
	方程式に帰着することを理解した上で,連立方程式を解くことができることをねらいとしている.
	また,数量の関係に注目し,連立方程式を利用して課題を解決することができることをねらい
	としている.
	\vspace{5mm}
		
	\item[3]指導について

		\hspace{1em}
		第1学年の方程式の単元で学習した一元一次方程式の解き方を念頭に置いて,それに帰着する
		ために加減法や代入法を用いて連立方程式を得必要がある.そのため,一次方程式の内容と
		異なる部分はどこか,それを解決するためにはどうすればよいか,常に問いかけながら,
		一元一次方程式に帰着させることに焦点をあてたい.連立方程式の利用では,身のまわりの
		課題について数量の関係に注目し,連立方程式を立てて課題を解決する活動を通して,
		連立方程式の有用性に気づかせたい.\\

		\hspace{1em}
		本学級の生徒は,授業では教師の指示を聞き,課題に熱心に取り組むことができる.
		しかし,文字式の計算を苦手としている生徒が多いため,段階を踏んで丁寧に進めていく.
		また,解放の手順を提示し,見通しを持って取り組めるようにしていく.

		\vspace{5mm}

	\item[4]	本時の目標
				文章から数量の関係を読み取り,連立方程式を立てることができる.
				\vspace{5mm}
	\item[5]
		本時の指導過程
	\end{enumerate}
	\vspace{-5mm}
		\begin{table}[htbp]
			\centering
			\hspace{5mm}
			\begin{tabular}{|p{0.5em}|>{\raggedright}p{17em}|>{\raggedright\arraybackslash}p{22em}|}
		\hline
		& & \tabularnewline
		& \multicolumn{1}{c|}{学習活動} &\multicolumn{1}{c|}{ 教師の支援(○)・評価(※)・協働(◇)}\tabularnewline
		\hline

		& & \tabularnewline

		 導入& 
			1 方程式の利用する手順を確認する.
				 &  \begin{enumerate}\vspace{-5mm}\item[○] 方程式を立てて解く手順を確認する.
		 \begin{enumerate}
			 \item[(1)]何が知りたい数か
			 \item[(2)]何と何が等しいか
			 \item[(3)]$x$を使って式をたてる
			 \item[(4)]方程式を解く
		 \vspace{-5mm}
		 \end{enumerate}
	 \end{enumerate}
		 \tabularnewline  
				 &	\multicolumn{2}{c|}{
					 \hspace{3mm}
					 \scalebox{0.8}{
						 \begin{itembox}[l]{復習}\large{ケーキ4個と150円のジュース1本買うとおつりは1550円でした.ケーキ1個はいくらですか?}
							\end{itembox}
							\hspace{3mm}
						}
					} \tabularnewline
					& & \begin{enumerate}\vspace{-5mm}
						\item[◇]  何が知りたいのか周りと相談させる.
					\end{enumerate}\tabularnewline
					\hline
			\end{tabular}
		\end{table}

\newpage
		\begin{table}[htbp]
			\centering
			\hspace{5mm}
	\begin{tabular}{|p{0.5em}|>{\raggedright}p{17em}|>{\raggedright\arraybackslash}p{22em}|}
\hline
		展開&\vspace{5mm}2 本時の目標を確認する &\tabularnewline

		& \multicolumn{2}{c|}{\fbox{文章から連立方程式を立てることができるようになろう} }\tabularnewline

		&3 課題を提示する & \tabularnewline
		& &\tabularnewline
				 &	\multicolumn{2}{c|}{
					 \hspace{3mm}
					 \scalebox{0.8}{
						 \toi{1}{
							 \large{1本100円のボールペンと1個150円の消しゴムを合わせて10個買うと1200円でした.それぞれいくつ買いましたか.
							}
						 }
							\hspace{3mm}
						}
					} \tabularnewline
					&& \begin{enumerate}
						\item[○] 復習の問題と異なる点を確認する.
						\item[○] 未知数を$x,y$と設定し,数量の関係に注目して式を立てさせる.
						\item[○] 解き終えた生徒には他の解放を考えさせる.
						\item[※] 数量関係から式を立てることができる.
							
							(ワークシート)
		 \vspace{-5mm}
					\end{enumerate} \tabularnewline

				 &	\multicolumn{2}{c|}{
					 \hspace{3mm}
					 \scalebox{0.8}{
						 \toi{2}{
							 \large{
							るレジャー施設の入園料は,おとな2人と小学生1人で4500円,
							おとな1人と小学生2人で3600円でした.おとなと小学生の入園料
							はそれぞれいくらですか?
							}
						 }
							\hspace{3mm}
						}
					} \tabularnewline
			& & \begin{enumerate}
						\item[※] 適切な未知数を文字でおくことができる.
						\item[※] 数量関係から式を立てることができる.
							
							(ワークシート)
		 \vspace{-5mm}
					\end{enumerate}\tabularnewline
			&4 練習問題 & ・ワークシートの問題を解かせる.\tabularnewline
					\hline

終末& \vspace{5mm} 5 まとめ & \begin{enumerate}
	\item[○] 連立方程式を利用して解く手順を確認する.
\end{enumerate}\tabularnewline
		& & \tabularnewline
			\hline
			\end{tabular}
		\end{table}
\begin{enumerate}
	\item[6] 評価の観点\\
		連立方程式を立てることに焦点化したことは,連立方程式を利用して課題を解決する手順を
		身につける上で有効だったか?
\end{enumerate}

\end{document}
