\RequirePackage{luatex85}
\documentclass[leqno]{ltjsarticle}% leqnoは数式番号左
\usepackage{luatexja-fontspec}
\usepackage[top=20truemm,bottom=20truemm,left=25truemm,right=25truemm]{geometry}
\usepackage{luatexja} 
\usepackage{multicol,amsmath,amssymb,mathtools,ascmac,amsthm,amscd,physics,comment,dcolumn,titlesec,mathrsfs,longtable,tabularx,mypkg}
\usepackage[all]{xy}
\titleformat*{\section}{\Large\bfseries}
\setlength{\parindent}{0pt}
\pagestyle{empty}
%\everymath{\displaystyle}
\begin{document}
{\center{{\textbf{\large{第2学年2組 数学科学習指導案}}}}
\begin{flushright}
	令和06年06月03日(月)第2校時\\
	場所 2年2組教室\\
	授業者\ \ \ 山本 雄大
\end{flushright}
\begin{enumerate}
	\item[1]
		単元(題材・主題)連立方程式
	\item[2] 単元(題材)について
	\item[3]
		指導について
	\item[4]
		本時の目標
		\begin{enumerate}
			\item[(1)]
		\end{enumerate}
	\item[5]
		本時の指導過程\\
		\begin{table}[htbp]
			\centering
			\begin{tabular}{|p{8em}|>{\raggedright}p{17em}|>{\raggedright\arraybackslash}p{17em}|}
			 \hline
		学習内容 & 生徒の学習活動 & 教師の支援(・)・評価(※)・協働(◇) \\
     & &   \\
   \hline \hline
    1 既習内容の確認 & ・復習問題を解く.\\(加減法,代入法) & ・早く終わった生徒に答えを小黒板に書かせる.  \\
		2 課題提示&・課題に取り組む. \[\left\{ \begin{array}{l} 5x - 3(2y+3x) = 0 \\ x+2y =-5 \end{array}\right.\] &  \tabularnewline
    3 &  &  \tabularnewline
    4 &  &  \tabularnewline 
    5&  & \tabularnewline
    6& &  \tabularnewline
    7& &  \tabularnewline
   \hline
		\end{tabular}
	\end{table}
	\item[6]
		評価の観点


\end{enumerate}

\end{document}

