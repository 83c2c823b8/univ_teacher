\RequirePackage{luatex85}
\documentclass[17pt,b4paper, landscape]{ltjsarticle}% leqnoは数式番号左
\usepackage{luatexja-fontspec}
\usepackage[top=10truemm,bottom=10truemm,left=10truemm,right=10truemm]{geometry}
\usepackage{luatexja} 
\usepackage{luacode,multicol,amsmath,amssymb,mathtools,ascmac,amsthm,amscd,physics,comment,dcolumn,titlesec,mathrsfs,mypkg}
\usepackage[all]{xy}
\titleformat*{\section}{\Large\bfseries}
\setlength{\parindent}{0pt}
\pagestyle{empty}

\begin{luacode*}
	function my_random(n)
		tex.sprint(math.random(n))
	end
\end{luacode*}

\newcommand*{\myRandom}[1]{
	\directlua{my_random(#1)}
}

%\newcommand*{\myRandoma}[1]{
%	\directlua{string.gsub(my_random(#1).."","1"," ")}
%}

\newcommand*{\renritu}{
	$\begin{cases}\myRandom{100}x  + \myRandom{100}y = \myRandom{100} \\
	\myRandom{100}x  + \myRandom{100}y = \myRandom{100} 
\end{cases}$\\
}
%\everymath{\displaystyle}
\setlength{\columnseprule}{0.4pt}
\begin{document}
{\textbf{\Large{連立方程式の利用}}}\hspace{\fill}{\scalebox{1.5}{( )組(        )}}\\
\begin{multicols*}{3}
	\renritu
	\renritu
	\renritu
	\renritu
	\renritu
	\renritu
	\renritu
	\renritu
\vfill\null\columnbreak
	\renritu
	\renritu
	\renritu
	\renritu
	\renritu
	\renritu
	\renritu
	\renritu
\vfill\null\columnbreak
	\renritu
	\renritu
	\renritu
	\renritu
	\renritu
	\renritu
	\renritu
	\renritu
\end{multicols*}

\end{document}
