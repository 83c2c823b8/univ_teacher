\RequirePackage{luatex85}
\documentclass[b4paper,16.5pt,paparesize, landscape]{ltjsarticle}% leqnoは数式番号左
\usepackage{luatexja-fontspec}
\usepackage[top=10truemm,bottom=10truemm,left=10truemm,right=10truemm]{geometry}
\usepackage{luatexja} 
\usepackage{multicol,amsmath,amssymb,mathtools,ascmac,amsthm,amscd,physics,comment,dcolumn,titlesec,mathrsfs,mypkg}
\usepackage[all]{xy}
\titleformat*{\section}{\Large\bfseries}
\setlength{\parindent}{0pt}
\pagestyle{empty}
%\everymath{\displaystyle}
\setlength{\columnseprule}{0.4pt}
\begin{document}
{\textbf{\Large{連立方程式の利用}}}\hspace{\fill}{\scalebox{1.5}{( )組(        )}}\\
\begin{multicols*}{3}
$(1)\begin{cases}
	x + y = 6\\ 2x + y = 10
\end{cases}$ 	\\[75mm]
$(2)\begin{cases}
	3x - 2y = 1\\ 6x -5y = -2
\end{cases}$\vfill\null\columnbreak

$(3)\begin{cases}
19x + 13y = 5\\ -3x + 15y = -69
\end{cases}$ 	\\[75mm]
$(4)\begin{cases}
	-7x + 4y = -37 \\ -20x + 3y  = -72
\end{cases}$
 \vfill\null\columnbreak
$(5)\begin{cases}
	-9x + 5y = -3 \\ 8x + 10y =46
\end{cases}$ 	\\[75mm]
$(6)\begin{cases}
	20x - 7y = -27\\ 7x + 16y =9
\end{cases}$



\end{multicols*}
\end{document}
