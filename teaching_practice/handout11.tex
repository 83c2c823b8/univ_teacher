\RequirePackage{luatex85}
\documentclass[b4paper,16.5pt,paparesize, landscape]{ltjsarticle}% leqnoは数式番号左
\usepackage{luatexja-fontspec}
\usepackage[top=10truemm,bottom=10truemm,left=10truemm,right=10truemm]{geometry}
\usepackage{luatexja} 
\usepackage{multicol,amsmath,amssymb,mathtools,ascmac,amsthm,amscd,physics,comment,dcolumn,titlesec,mathrsfs,mypkg}
\usepackage[all]{xy}
\titleformat*{\section}{\Large\bfseries}
\setlength{\parindent}{0pt}
\pagestyle{empty}
\everymath{\displaystyle}
\setlength{\columnseprule}{0.4pt}
\begin{document}
{\textbf{\Large{加減法の演習}}}\hspace{\fill}{\scalebox{1.5}{( )組(        )}}\\
\begin{multicols*}{3}
$(1)\begin{cases}
	2x + 3y = 1 \\
	5 x + 7y = 2
	% (-1,1)
\end{cases}$ 	\\[82mm]
(2)$\begin{cases}
	-7x + 12y = 2\\
	4x + 5y = -13
	% (-2,-1)
\end{cases}$\vfill\null\columnbreak

$(3)\begin{cases}
	4x + 8y = 12\\
	3x - 11y = -24
	% (-3,3)
\end{cases}$ 	\\[82mm]

$(4)\begin{cases}
	2x + 3y = 5\\
	5x + 7y = 11
	% (-2,3)
\end{cases}$
 \vfill\null\columnbreak

$(5)\begin{cases}
	-7x - 5y = -11\\
	2x + 4y = 7
	% (1/2,3/2)
\end{cases}$ 	\\[82mm]

$(6)\begin{cases}
	3x + 5y = 7\\
	11x + 13 = 17
	% (-3/8 , 13/8)
\end{cases}$



\end{multicols*}
\end{document}
