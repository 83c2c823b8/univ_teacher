\RequirePackage{luatex85}
\documentclass[b4paper,16.5pt,paparesize, landscape]{ltjsarticle}% leqnoは数式番号左
\usepackage{luatexja-fontspec}
\usepackage[top=10truemm,bottom=10truemm,left=10truemm,right=10truemm]{geometry}
\usepackage{luatexja} 
\usepackage{multicol,amsmath,amssymb,mathtools,ascmac,amsthm,amscd,physics,comment,dcolumn,titlesec,mathrsfs,mypkg}
\usepackage[all]{xy}
\titleformat*{\section}{\Large\bfseries}
\setlength{\parindent}{0pt}
\pagestyle{empty}
\everymath{\displaystyle}
\setlength{\columnseprule}{0.4pt}
\begin{document}
{\textbf{\Large{加減法の演習}}}\hspace{\fill}{\scalebox{1.5}{( )組(        )}}\\
\begin{multicols*}{3}
$(1)\begin{cases}
	2x + 3y = 1 \\
	3 x + 7y = 4
	% (-1,1)
\end{cases}$ 	\\[82mm]
(2)$\begin{cases}
	-3x + 2y = 4\\
	4x - 5y = 3
	% (-2,-1)
\end{cases}$\vfill\null\columnbreak

$(3)\begin{cases}
	-4x + 8y = -8\\
	3x - 7y = 9
	% (-2,-2)
\end{cases}$ 	\\[82mm]

$(4)\begin{cases}
	3x + 2y = -1\\
	7x + 5y = -3
	% (1,-2)
\end{cases}$
 \vfill\null\columnbreak

$(5)\begin{cases}
	2x + 3y = 5\\
	7x + 11y = 13
	% (16,-9)
\end{cases}$ 	\\[82mm]

$(6)\begin{cases}
	-5x - 7y = -11\\
	4x + 2y = 7
	% (3/2,1/2)
\end{cases}$



\end{multicols*}
\end{document}
