\RequirePackage{luatex85}
\documentclass[leqno]{ltjsarticle}% leqnoは数式番号左
\usepackage{luatexja-fontspec}
\usepackage[top=5truemm,bottom=5truemm,left=25truemm,right=25truemm]{geometry}
\usepackage{luatexja} 
\usepackage{multicol,multirow,amsmath,amssymb,mathtools,ascmac,amsthm,amscd,physics,comment,dcolumn,titlesec,mathrsfs,longtable,tabularx,mypkg}
\usepackage[all]{xy}
\titleformat*{\section}{\Large\bfseries}
\setlength{\parindent}{0pt}
\pagestyle{empty}
%\everymath{\displaystyle}
\begin{document}
%{\center{{\textbf{\large{第2学年2組 数学科学習指導案}}}}
\begin{flushright}
	令和06年06月10日(水)第2校時\\
	場所 2年2組教室\\
	授業者\ \ \ 山本 雄大
\end{flushright}
\begin{table}[htbp]
	\begin{tabular}{|p{8em}|>{\raggedright}p{17em}|>{\raggedright\arraybackslash}p{17em}|}
		\hline
		学習内容 & 生徒の学習活動 & 教師の支援(・)・評価(※)・協働(◇) \\
     & &   \\
   \hline \hline

1		復習 & &・プリント配布する.(15分)\tabularnewline
	& & \tabularnewline
		\multicolumn{3}{|c|}{
		\toi{}{\centering
			(1)$ \begin{cases}{}
				x + 2y = 3\\
				x + 3y = 5\\
				(-1,2)
			\end{cases} $ 
			(2)$\begin{cases}{}
				3x - 5y = 13\\
				x + 7y = -13\\
				(1,-2)
			\end{cases} $
		}
	}\tabularnewline


	2 確認& ・解き方の手順を確認する.\\
	(Step 1) 片方の式を整数倍して\underline{係数}の絶対値を揃える.\\
	(Step 2)足すか引くかして文字を\underline{消去}する. \\ 
	(Step 3)解く!\\ 
	(Step 4)\underline{代入}して解く.\\
	(Step 5)確かめる.
		& \begin{tabular}{l}
			・穴埋めさせる.\\
			◇生徒同士で確認する\\
			・具体例(4)と比較しつつ説明する.
		\end{tabular}\tabularnewline

	3 課題の提示 & & \tabularnewline
	 & & \tabularnewline
	\multicolumn{3}{|c|}{
		\toi{}{\centering
			(5)$ \begin{cases}{}
			4x + 3y = -1\\
			3x + 7y = -15\\
			(2,-3)
		\end{cases}$
	}
}\tabularnewline
& 解けなかったら「どこが違う?」\\
「どうしたら解ける?」& \tabularnewline

	\multicolumn{3}{|c|}{\fbox{加減法でそのまま足したり,引いたりしても文字が消去できない場合にとけるようになる.}}\tabularnewline

& & \tabularnewline
	4 本時の目標の確認 & & \tabularnewline
&"(Step 1) どちらかの式を整数倍する"\\
	ができないのでどうすれば,目的の消去したい文字の係数の絶対値をそろえることができるか考える.
	\\ \rightarrow 両方整数倍すれば係数の?絶対値を揃えることができる.(有理数倍?) & ◇周りの人と相談する.\tabularnewline
																																															& & ・それ以外のStepは同様に進むことができることを伝える.\tabularnewline

	& & \tabularnewline
	5 練習問題 & 問題演習   &  \tabularnewline 
	& & \tabularnewline
	& & \tabularnewline
		\hline
	\end{tabular}
\end{table}

\end{document}}
