\RequirePackage{luatex85}
\documentclass[leqno]{ltjsarticle}% leqnoは数式番号左
\usepackage{luatexja-fontspec}
\usepackage[top=20truemm,bottom=20truemm,left=25truemm,right=25truemm]{geometry}
\usepackage{luatexja} 
\usepackage{multicol,multirow,amsmath,amssymb,mathtools,ascmac,amsthm,amscd,physics,comment,dcolumn,titlesec,mathrsfs,longtable,tabularx,mypkg}
\usepackage[all]{xy}
\titleformat*{\section}{\Large\bfseries}
\setlength{\parindent}{0pt}
\pagestyle{empty}
%\everymath{\displaystyle}
\begin{document}
{\center{{\textbf{\large{第2学年2組 数学科学習指導案}}}}
\begin{flushright}
	令和06年06月03日(月)第2校時\\
	場所 2年2組教室\\
	授業者\ \ \ 山本 雄大
\end{flushright}
\begin{table}[htbp]
	\begin{tabular}{|p{8em}|>{\raggedright}p{17em}|>{\raggedright\arraybackslash}p{17em}|}
		\hline
		学習内容 & 生徒の学習活動 & 教師の支援(・)・評価(※)・協働(◇) \\
     & &   \\
   \hline \hline

		1 本時の目標の確認 & & \tabularnewline
	\multicolumn{3}{|c|}{\fbox{二元一次連立方程式をだいたい解ける.}}\tabularnewline
2		復習 & &15分ストップウォッチではかる.\tabularnewline
	& & \tabularnewline
	& & \tabularnewline
		\multicolumn{3}{|c|}{
		\toi{}{
			$\begin{cases}{}
				x + 2y  = 3\\
				x + 3y = 5\\
				(-1,2)
			\end{cases} $ 
			$\begin{cases}{}
				9x + 4y = -15\\
				7x - 4y = -26\\
				(-2,3)
			\end{cases} $
			$\begin{cases}{}
				3x  - 5y = 13\\
				x + 7y = -13\\
				(1,-2)
			\end{cases} $
			$\begin{cases}{}
				-7x - 8y = 3\\
				14x - 5y = 57\\
				(3,-3)
			\end{cases} $
		}
	}\tabularnewline
	3& ・解き方を確認する.\\
	(1) 係数の符号が一方の整数倍になっているものを見つける.\\
	(2) \\ 
	(3) \\ 
	(4) 係数

		&\tabularnewline
3課題提示

	& & \tabularnewline

	& & \tabularnewline

	& & \tabularnewline
4練習問題 &ワークシートの問題を解く.   &  \tabularnewline 
	& & \tabularnewline
	& & \tabularnewline
		\hline
	\end{tabular}
\end{table}

\end{document}}
